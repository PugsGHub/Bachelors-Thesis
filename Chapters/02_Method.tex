\chapter{Analytical Mechanics}
\label{ch:method}
\section{Lagrangian Mechanics}
\begin{tcolorbox}
    For
\end{tcolorbox}
field theory lagrangian
the EOM are obtained when we stationarize the the action i.e. equation the Frechet derivative to zero
\subsection{Constraints}
\subsection{Lagrange Multipliers}
\begin{center}
\begin{tabularx}{0.99\textwidth} { 
		| >{\raggedright\arraybackslash}X 
		| >{\centering\arraybackslash}X 
		| >{\raggedleft\arraybackslash}X | }
	\hline
\textbf{Characteristic of Intertial Frame} & \textbf{Property of Lagrangian} & \textbf{Conserved Quantity} \\
	\hline
	Time homogeneous & Not explicit function of time & Total energy \\
	\hline
	Time homogeneous   & Invariant to translation  & Linear momentum  \\
	\hline
	Space isotropic   & Invariant to rotation  & Angular momentum  \\
	\hline
\end{tabularx}
			\end{center}
\section{Hamilton's Mechanics}
Hamilton's equations are written as,
\begin{tcolorbox}
	\begin{equation}
\dot{q}_{i} = \frac{\partial H}{\partial p_{i}}
\end{equation}
\begin{equation}
-\dot{p}_{i} = \frac{\partial H}{\partial q_{i}}
\end{equation}
\end{tcolorbox}
	
	Where,
	$$H = T + V = \sum_{i = 1}^{n} p_{i}q_{i} - \mathcal{L}$$
	$$P = \frac{\partial \mathcal{L}}{\partial \dot{q}}$$
\subsection{Modified Hamilton's Principle}
We can thus modify Hamilton's principle to incorporate the Hamiltonian as,
\begin{equation}
\delta \int_{t_{1}}^{t_{2}} \left(\sum_{i = 1}^{n} p_{i}q_{i} - H \right) dt = 0 
\end{equation}
\subsection{Poisson Brackets}
We define Poisson Brackets as,
		\begin{equation}
		\{A, B\} = \sum_{i}^{n} \left( \frac{\partial A}{\partial q_{i}} \frac{\partial B}{\partial p_{i}} -  \frac{\partial B}{\partial q_{i}} \frac{\partial A}{\partial p_{i}}\right)
		\end{equation}
		They have the following properties,
		\begin{itemize}
		\item \textbf{Antisymmetry:} $\{A, C\} = -  \{C, A\}$
		\item \textbf{Bilinearity:} $\{kA, C\} = k \{A, C\}$
		\item $\{\left(AB\right), C\} = B\{A,C\} + A\{B,C\}$
		\end{itemize}
		
		We can rewrite Hamilton's equations through Poisson Brackets as,
		\begin{equation}
		\dot{p}_{i} = \{p_{i}, H\}
		\end{equation}
		\begin{equation}
		\dot{q}_{i} = \{q_{i}, H\}
		\end{equation}
		\begin{equation}
			\{q_{i}, p_{j}\} = \delta_{ij}
		\end{equation}
\section{Some Niche Stuff}
\subsection{Liouville's Theorem}
\subsection{Virial Theorem}
\begin{equation}
	S= \sum_{i}^{n} p_{i} . r_{i}
	\end{equation}
	\begin{equation}
	\frac{d S}{d t} = \sum_{i}^{n} \dot{p}_{i} . r_{i} + p_{i} . \dot{r}_{i}
	\end{equation}
	\begin{equation}
	\expectationvalue{\frac{d S}{d t}} = \frac{1}{\tau} \int_{0}^{r} \frac{d S}{d t} dt = \frac{S(\tau) - S(0)}{\tau}
	\end{equation}
	
	\begin{equation}
		\expectationvalue{\sum_{i}^{n} p_{i} . \dot{r}_{i}} = - \expectationvalue{\sum_{i}^{n} \dot{p}_{i} . r_{i}}
		\end{equation}
		\begin{equation}
		\expectationvalue{2 \sum_{i}^{n} T_{i}} = - \expectationvalue{\sum_{i}^{n} \dot{F}_{i} . r_{i}}
		\end{equation}
		\begin{equation}
		\expectationvalue{T} = - \frac{1}{2} \expectationvalue{\sum_{i}^{n} \dot{F}_{i} . r_{i}}
		\end{equation}
\section{Noether's Theorem}
\subsection{The Rund-Tutman Identity}
